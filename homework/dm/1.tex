
%HSE Template
\documentclass[a4paper,12pt]{article}

\usepackage[unicode,colorlinks=true,linkcolor=blue]{hyperref}
\usepackage{amsmath,amssymb}
\usepackage[utf8]{inputenc}
\usepackage[T2A]{fontenc}
\usepackage[russian]{babel}
\usepackage{graphicx}
\usepackage[margin=1in]{geometry}
\usepackage{fancyhdr}

\pagestyle{fancy}
\makeatletter
\fancyhead[L]{\footnotesize ШАД, 2013/14, <<Дискретная математика>>}
\fancyfoot[L]{\footnotesize \@author}
\fancyfoot[R]{\thepage}
\fancyfoot[C]{}

\renewcommand{\maketitle}{%
\noindent{\bfseries\scshape\large\@title\ \mdseries\upshape(\@date)}\par
\noindent {\large\itshape\@author}
\vskip 2ex}
\makeatother

\newenvironment{problem}[1]{\par\bigskip\noindent\textbf{Решение задачи #1.}
  \enskip\ignorespaces}{}

\title{Решение домашнего задания №1} % Fill the number of the homework
\author{Тропин А. Г. \\
  e-mail: \href{mailto:andrewtropin@gmail.com}{andrewtropin@gmail.com} \\
  github: \href{http://github.com/abcdw/}{abcdw}}
%\date{12 сентября 2013 г.} % Fill the date

\begin{document}
  \maketitle
  \thispagestyle{fancy}

  \begin{problem}{1}
    Черную клетку можно выбрать 32 способами, остается (32 - 8) подходящих
    белых клеток. И того способов выбрать две клетка 32*24. \\

    Ответ: 768.
  \end{problem}

  \begin{problem}{2}
    Заметим, что $C_9^k$ - число таких чисел, но с фиксированной длиной k.
    $\{a_1, a_2, \dots, a_k\}, a_i = a_j \Rightarrow i = j$, без ограничения
    общности можно считать, что $a_1 < a_2 < \dots < a_k$. Еще одно число это
    0, которое мы не где не учли. \\

    Ответ: $\sum\limits_{i = 1}^{6} C_9^i + 1 = 466$
  \end{problem}

  \begin{problem}{3}
    \\

    Ответ:
    \begin{enumerate}
      \item $10^2 * 12 * 10^3 * 12^2 = 172800000$
      \item $3 * 12 * 10^3 * 12^2 = 5184000$
    \end{enumerate}
  \end{problem}

  \begin{problem}{4}
    Элементы независимы 3 * 7 * 7. \\

    Ответ: 147.
  \end{problem}

  \begin{problem}{5}
    \\

    Ответ:
    \begin{enumerate}
      \item 5*5 = 25
      \item 5*4 = 20
    \end{enumerate}
  \end{problem}

  \begin{problem}{6}
    Берем 12 открыток и расставляем между ними 9 перегородок, разбивая тем самым
    их на 10 типов. Вариантов поставить пергородки~---~$C_{21}^9$. \\

    Ответ: $C_{21}^9$.
  \end{problem}

  \begin{problem}{7}
    Всего трупп --- $C_{10}^6$. Способов выбрать --- $C_{10}^6$ в первый день
    и на единицу меньше каждый последующий и так в течении 14 дней. \\

    Ответ: $\frac{(C_{10}^6)!}{(C_{10}^6 - 14)!}$.
  \end{problem}

  \begin{problem}{8}
    $C_{2n-1}^n$ --- количество слов , $n C_{2n-1}^n$ --- количество букв
    в этих словах. Всего $n$ различных букв. Все буквы ``симметричны''.
    Поэтому общее число появлений каждой буквы будет
    $\frac{n C_{2n-1}^n}{n}$.

    Ответ: $C_{2n-1}^n$.
  \end{problem}

  \begin{problem}{9}
    \begin{enumerate}
      \item Будем задавать квадрат координатой левого верхнего угла и длинной
        стороны. Для длины стороны $l$, есть $n-l+1$ координат $x$ и $n-l+1$
        координат $y$. И того $(n-l+1)^2$ положений левого верхнего угла
        квадрата со стороной $l$. \\

        Ответ: $\sum\limits_{l = 1}^{n} (n-l+1)^2$
      \item Будем задавать прямоугольника координатами двух несоседних вершин,
        первую вершину мы можем выбрать $(n+1)^2$ способами, вторую $n^2$
        способами. Плюс к тому $[a,b,c,d] = [c,d,a,b] = [c,b,a,d] = [a,d,c,b]$,
        значит нужно еще разделить на 4. \\

        Ответ: $\frac{n^2(n+1)^2}{4}$

      \item Будем задавть аналогично прямоугольнику, за тем исключением, что
        у нас есть минимальный размер и фигура имеет ориентацию, то есть те
        4 пункта, которые были одинаковы в предыдущем пункте, теперь станут
        различны. Сделаем ``рамочку'' справа 2, снизу 1 клетки, чтобы решить
        проблему минимального размера. Мы будем задавать прямоугольник и
        пририсовывать к правому нижнему углу еще один прямоугольник размера
        $1\times2$. Задача сводится к получению количества прямоугольников в
        квадрате с ``рамочкой'', то есть прямоугольнике $(n-1)\times (n-2)$
        и умножению его на 4. \\

        Ответ: $\frac{(n-1)(n-2)(n-2)(n-3)}{4}4 = (n-1)(n-2)^2(n-3)$
    \end{enumerate}
  \end{problem}

  \begin{problem}{10}
    $S_p$ --- суммарная площадь попарных пересейчений. $5 \geq 9 * 1 - S_p$.
    Всего $8*9/2$ попарных пересечений. Предположим что, ни одно персечение не
    больше $\frac{1}{9}$, получаем, что площадь прямоугольника
    $> 9 - \frac{1}{9} 36 = 5$
  \end{problem}

  \begin{problem}{11}
    При выборе одной точки на одной стороне точки, на двух оставшихся уже
    нельзя выбирать одну точку, чтобы не было параллельной стороны. Выбираем
    вторую точку на второй стороне и на третей уже две точки, которые нельзя
    брать. \\

    Ответ: $7*6*5 = 210$.
  \end{problem}

  \begin{problem}{13}
    Пусть все сапоги 41ого --- правые, 42ого --- левые, 43ого --- парные(100
    пар). С помощью несложных махинаций мы можем получить любую конфигурацию.
    К примеру берем правый сапог 41ого и ``превращаем'' его в правый 43ого,
    а соответственно левый 43его в левый 41ого. И количество пар всегда не
    меньше 100.
  \end{problem}

  \pagebreak
  \begin{problem}{14}
    12 можно набирать двойками или одной двойкой и двумя пятерками. Если
    набираем шестью двойками, то получаем $C_{2013}^6$, коэффициент
    положительный, так как четное число $-x^2$. Две пятерки можно выбрать
    $C_{2013}^2$ и остается 2011 скобок из которых можно взять двойку,
    коэффициент будет отрицательным, так как
    $1 * 1 * \cdots * 1 * -x^2 * -x^5 * -x^5$. \\

    Ответ: $C_{2013}^6 - 2011C_{2013}^2$.
  \end{problem}

  %\begin{problem}{15}
  %\end{problem}
\end{document}
