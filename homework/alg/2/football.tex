%HSE Template
\documentclass[a4paper,12pt]{article}

\usepackage[unicode,colorlinks=true,linkcolor=blue]{hyperref}
\usepackage{amsmath,amssymb}
\usepackage[utf8]{inputenc}
\usepackage[T2A]{fontenc}
\usepackage[russian]{babel}
\usepackage{graphicx}
\usepackage[margin=1in]{geometry}
\usepackage{fancyhdr}

\pagestyle{fancy}
\makeatletter
\fancyhead[L]{\footnotesize 2013/14, <<Алгоритмы и структуры данных поиска>>}
\fancyfoot[L]{\footnotesize \@author}
\fancyfoot[R]{\thepage}
\fancyfoot[C]{}

\renewcommand{\maketitle}{%
\noindent{\bfseries\scshape\large\@title\ \mdseries\upshape(\@date)}\par
\noindent {\large\itshape\@author}
\vskip 2ex}
\makeatother

\newenvironment{problem}[1]{\par\bigskip\noindent\textbf{Решение задачи #1.}
  \enskip\ignorespaces}{}

\title{Решение домашнего задания} % Fill the number of the homework
\author{Тропин А. Г. \\
  e-mail: \href{mailto:andrewtropin@gmail.com}{andrewtropin@gmail.com} \\
  github: \href{http://github.com/abcdw/}{abcdw}}
\date{22 октября 2013 г.} % Fill the date

\begin{document}
  \maketitle
  \thispagestyle{fancy}

  \textbf{Футболисты.}

  Возьмем двух любых человек, с эффективностями $e_1$ и $e_2$ соответственно,
  если они принадлежат идеальной команде, то и люди с эффективностями в
  промежутке $(e_1, e_2]$ тоже принадлежат команде. Тогда логично посортить
  футболистов по эффективности и искать самых левого и правого из футболистов,
  принадлежащих идеальной команде.

  Теперь мы можем задавать команду левым и правым футболистом, изначально
  возьмем команду из одного человека $l = 0, r = 0$. Заметим, что пока мы
  увеличиваем $r$ и выполняется условие $e_l + e_{l + 1} >= e_r$, то суммарная
  эффективность только увеличивается, поэтому при увеличении $r$ мы просто
  к суммарной эффективности прибавляем эффективность очередного футболиста.
  Как только условие нарушается, мы начинаем увеличивать левую границу.
  
  Почему перебирая таким образом мы не пропустим ни одной команды, которая
  может быть лучше чем те, что мы перебрали? Потому что любая другая команда
  будет состоять из подмножества игроков одной из перебранных команд(мы для
  каждого $l$ берем максимально правый $r$). \\

  {\itshape{Асимптотика:}}

  Сортировка работает за $O(n \log n)$. Каждая из границ меняется не больше
  чем $n$ раз, всего две границы ($l,r$), значит всего будет не больше
  $2n = n + n$ сдвигов границ. За каждый сдвиг границы мы пересчитываем сумму
  и сравниваем ее с текущей максимальной за $O(1)$. И того алгоритм без
  сортировки работает за $O(n)$, с сортировкой соответственно за $O(n \log n)$

\end{document}
