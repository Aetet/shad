%HSE Template
\documentclass[a4paper,12pt]{article}

\usepackage[unicode,colorlinks=true,linkcolor=blue]{hyperref}
\usepackage{amsmath,amssymb}
\usepackage[utf8]{inputenc}
\usepackage[T2A]{fontenc}
\usepackage[russian]{babel}
\usepackage{graphicx}
\usepackage[margin=1in]{geometry}
\usepackage{fancyhdr}

\pagestyle{fancy}
\makeatletter
\fancyhead[L]{\footnotesize 2013/14, <<Алгоритмы и структуры данных поиска>>}
\fancyfoot[L]{\footnotesize \@author}
\fancyfoot[R]{\thepage}
\fancyfoot[C]{}

\renewcommand{\maketitle}{%
\noindent{\bfseries\scshape\large\@title\ \mdseries\upshape(\@date)}\par
\noindent {\large\itshape\@author}
\vskip 2ex}
\makeatother

\newenvironment{problem}[1]{\par\bigskip\noindent\textbf{Решение задачи #1.}
  \enskip\ignorespaces}{}

\title{Решение домашнего задания} % Fill the number of the homework
\author{Тропин А. Г. \\
  e-mail: \href{mailto:andrewtropin@gmail.com}{andrewtropin@gmail.com} \\
  github: \href{http://github.com/abcdw/}{abcdw}}
\date{29 октября 2013 г.} % Fill the date

\begin{document}
  \maketitle
  \thispagestyle{fancy}

  \textbf{Менеджер памяти.}

  Нам нужно на каждом вопросе о выделении памяти уметь быстро искать
  самый свободный кусок памяти, поэтому заведем кучу, она как раз для этого
  неплохо подходит. Будем хранить в куче все свободные участки памяти,
  за логарифм мы вытаскиваем самый свободный, от него пытаемся отрезать кусок,
  если получается, то говорим, где начинается этот участок, образеам свободный
  участок и запихиваем обратно в кучу, если не получается, говорим -1.

  С вставкой разобрались, теперь нужно реализовать удаление. У нас может быть
  несколько различных ситуаций. 
  \begin{itemize}
    \item $[\_\_][XX][\_\_] \to [\_\_\_\_\_\_]$
    \item $[**][XX][\_\_] \to [**][\_\_\_\_]$
    \item $[\_\_][XX][**] \to [\_\_\_\_][**]$
    \item $[**][XX][**] \to [**][\_\_][**]$
  \end{itemize}

  Будем хранить все интервалы памяти в двусвязанном списке(начало интервала,
  количество свободной или занятой памяти и положение в куче). Теперь для
  отмены операции выдиления памяти, нам нужно найти, этот участок памяти в
  списке, можем делать это за константу(храним в операции итератор на элемент
  списка), ну тогда логично хранить в куче только итераторы на элементы
  списка и по ним обновлять положение в куче при проталкивании элемента.
  Теперь мы можем выполнять слияния, как показано в четырех пунктах выше, 
  с помощью операциями над списком, при этом мы будем поддерживать актуальное
  состояние кучи. \\

  {\itshape{Асимптотика:}}

  При выделении элемента мы один раз обращаемся к списку $O(1)$ и два раза
  к куче $O(\log M)$, при освобождении памяти мы максимум обращаемся к трем
  элементам списка и делаем максимум две операции удаления $O(1)$ и максимум
  два обращения к куче $O(\log M)$. Всего $M$ запросов и того асимптотика
  $O(M \log M)$.



\end{document}
